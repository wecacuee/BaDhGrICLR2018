\paragraph{Localization and mapping}
Robotic localization and mapping for navigation is a classic problem in  mobile robotics and sensing.
\cite{SmChIJRR1986} introduced the idea of propagating spatial uncertainty for robot localization while mapping and \cite{ElCOMPUTER1980} popularized Occupancy Grids.
In the three decades since these seminal works, the field has exploded with hundreds of algorithms released for multiple kinds of sensors such as cameras, laser scanners, sonars and depth sensors.
These algorithms vary in how much detail is captured in the map. For example, topological maps, like \cite{KuCOGSCI1978}, aim to capture as little information as possible while occupancy grid maps, \cite{ElCOMPUTER1980}, aim to capture metrically accurate maps in resolutions dependent upon the navigation task.

% JJC: tie it back to the main theme of the paper
All these approaches require significant hand-tuning depending on the environment, sensor types and navigation constraints of the hardware.
%The level of detail utilized in the creation of the  maps also needs to be decided before hand irrespective of the application and hence the amount of information stored is not optimized for the task at hand.
In contrast, end-to-end navigation algorithms optimize the detail of map storage based on the navigation task at hand which makes them worth exploring.

\paragraph{Deep reinforcement learning}
%Deep reinforcement learning (DRL) was originally conceived in 1995 (\cite{TeACM1995}).
DRL gained prominence recently when \cite{MnKaSiNIPSDLW2013,MnKaSiNATURE2015} used DRL to create agents that outperformed humans on Atari games while being purely trained on the games visual outputs.
% VD: On a second look, this train of thought is a distraction
%Subsequently, the field has been extended in several directions \cite{MnBaMiICML2016} being applied to creation of the Alpha-GO agent\cite{SiHuMaNATURE2016}, simulated platforms \cite{KaStJoNIPS2017}, real world robots \cite{LePaKrISER2017} and more recently to robotic navigation \cite{MiPaViICLR2017,OhChSiICML2016}.
Recently, DRL has been applied to end-to-end navigation (\cite{OhChSiICML2016,MiPaViICLR2017,ChLaSaNIPS2016}).
%The exploration into robotic navigation using DRL is a nascent topic having the potential to disrupt the fields of simultaneous localization, mapping and path planning.
% VD: this is repeated in the next line
%However literature concerning DRL based navigation usually ignores the standard machine learning practice of separating the training and testing sets thereby limiting our understanding of the generality of these methods.
In fact, it is common for agents to be trained and tested on the same maps with the variation only being applied to the agent's initial spawn point and the map's goal location (\cite{MiPaViICLR2017,ZhMoKoICRA2017,KuSaGaAPA2016}). 

In contrast, \cite{OhChSiICML2016} do test their algorithm on random unseen maps but their agents are trained to choose between multiple possible goal locations based on past observations.
The episodes end when the agent collects the goal, so there is no requirement for the algorithm to store map information during their exploration.
Thus the decisions made by these agents are to avoid a goal of a particular color and seek other colors rather than remembering the path to the goal.
On similar lines, \cite{ChLaSaNIPS2016} test their method on unseen maps in the VizDoom environment but only vary the maps with unseen textures. Thus their agents are texture invariant but continue to train and test on maps with the same geometric structure.
%
% TODO: Add more literature. Who has cited \cite{MiPaViICLR2017} in the last one year and address the latest developments.

In this work, we extend the study of these methods in a more comprehensive set of experiments to address the question of whether DRL-based agents remember enough information to obviate mapping algorithms or whether they may in fact need to be augmented with them for future progress.
 
