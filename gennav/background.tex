Our problem formulation is based upon the work of \cite{MiPaViICLR2017}. We summarize the technical setup here for completeness.
We refer the reader to \cite{MnBaMiICML2016,MiPaViICLR2017} for more details of the setup.

The problem of navigation is formulated as interaction between an environment and an agent.
At any time $t$ the agent takes an action $\actt \in \Action$ and observes observation $\obst \in \Obs$ along with a reward $\rewt \in \R$.
We assume the environment to be Partially Observable Markov Decision Process (POMDP).
In a POMDP, the state of the environment $\statet \in \State$
is assumed to be the only information that is propagated over time and both
$\obst$ and $\rewt$ are assumed to be independent of previous states given current state and last action.
Formally, a POMDP is a six tuple $(\Obs, \ObsFunc, \State, \Action, \Trans, \Rew)$ that is observation space $\Obs$, observation function $\ObsFuncFull$, state space $\State$, action space $\Action$, transition function $\TransFull$ and reward function $\RewFull$ respectively.
For our problem setup, the observation space $\Obs$ is the space of encoded feature vector that can be generated from input image or combination of other inputs, action space $\Action$ contains four actions: rotate left, rotate right, moved forward and move backward and reward function $\Rew$ is defined for each experiment so that the reaching the goal leads to high reward with auxilary reward to encourage certain kind of behavior.

For Deep reinforement learning the state space $\State$ is not hand tuned, instead it is modeled as semantically meaningless \emph{hidden state} of a fixed size float vector.
Also, instead of modeling observation function $\ObsFuncFull$ and $\TransFull$, a combined transition function $\TransObsFull$ is modeled such that it estimates next state $\statetn$ directly considering previous observation as well as reward into account. For policy-based DRL a policy function $\piDef$ and a value function $\ValueDef$ are also modeled. All three functions $\TransObs$, $\pit$, $\Valuet$ share most of the parameters in a way such that $\theta_T \subseteq \theta_{\pi} \cap \theta_\Value$

%Since we aim to estimate $\ObsFuncInv$ and $\Trans$ from experience, we formulate the experience as observation tuples divided into episodes of fixed length $E$.
%Each episode experience contains $E$ tuples with observation, action and corresponding reward $D_E = \{(\obs_0, \act_0, r_0), \dots, (\obs_E, \act_E, r_E)\}$. After each episode the state $\state_t$ is reset to all zeros and another data sequence is collected. Let the collected dataset be $D_N = \{
Our objective is to estimate unknown weights $\theta = \theta_T \cup \theta_\pi \cup \theta_V$ that maximizes the expected future reward, $R_t = \sum_{k=t}^{t_{end} - t} \gamma^{k-t} r_k$, where $\gamma$ is the discount factor,
%
\begin{align}
\theta^* = \arg\max_{\theta} \E[R_t] \,.
\end{align}
%
% need \graphicspath{{images/}}
\def\svgwidth{0.25\columnwidth}%
\begin{figure}%
\input{images/a3c-as-pomdp.pdf_tex}%
\def\svgwidth{0.25\columnwidth}%
\input{images/a3c-as-nn.pdf_tex}%
\usetikzlibrary{positioning}
\begin{tikzpicture}
\node[networknode,fill=green!20] at (10.5, 2.5) (lstm2) {LSTM:64};
\node[networknode,fill=green!20,left=0.6 of lstm2] (lstm) {LSTM:256};
\node[networknode,fill=orange!20,left=0.8 of lstm]  (enc2) {\parbox{8ex}{CNN:32\\4x4/2x2}};
\node[networknode,,fill=orange!20,left=0.4 of enc2]  (enc) {\parbox{8ex}{CNN:16\\8x8/4x4}};
\node[left=0.4 of enc, inputvar] (It) {$I_t$:84x84x3};
\node[above=0.1 of It,inputvar] (at) {$\acttp$};
\node[below=0.1 of It,inputvar]  (rt) {$\rewtp$};
\node [right=0.8 of lstm2,outvar] (V)  {$V$, $\pi$};
\node [shift={(-0.4,0)},above=0.4 of V,auxvar] (pi)  {$L$};
\node [shift={(-0.4,0)},below=0.4 of V,auxvar] (D)  {$D_2$};
\draw [-stealth] (It) edge (enc);
\draw [-stealth] (at) edge [in=110,looseness=0.4] (lstm2);
\draw [-stealth] (enc2) edge [bend left=70,looseness=0.4] (lstm2);
\draw [-stealth] (enc) edge (enc2);
\draw [-stealth] (enc2) edge node [shift={(0.225,0.1)},left]{$o_t$} (lstm);
\draw [-stealth] (rt) edge  [bend right,looseness=0.3] (lstm);
\draw [-stealth] (lstm) edge node[below=0.4,auxvar] (D1) {$D_1$} (lstm2);
\draw [-stealth] (lstm) edge [bend left] (D1);
\draw [-stealth] (lstm2) edge [bend right] (pi);
\draw [-stealth] (lstm2) edge  (V);
\draw [-stealth] (lstm2) edge [bend left] (D);
%\draw [use as bounding box] (5.0,1.5) rectangle (12.0, 4.0);
\end{tikzpicture}
\hspace{-1ex}%
\begin{tikzpicture}
%\draw [use as bounding box] (0,0) rectangle (1.8in, 3in);
\node[] at (2.3, 2) (It) {$I_t$: 84x84x3};
\node[networknode, above=1.2 of It, text width=45, align=center] (c1) {conv\\
  3x3/2x2,32};
\node[above=1.2 of c1] (ft) {$o_t$: 256};
\draw [-stealth] (c1) edge [out=30, in=-30,looseness=2] node [right] {4x}(c1);
\draw [-stealth] (It) -- (c1);
\draw [-stealth] (c1) -- (ft);
\end{tikzpicture}
\hspace{-1ex}%
\begin{tikzpicture}
%\draw [use as bounding box] (0,0) rectangle (1.8in, 3in);
\node[] at (2.3, 2) (It) {$I_t$: 84x84x3};
\node[networknode, above=1.2 of It] (fc) {FC(256)};
\node[above=1.2 of fc] (ft) {$o_t$: 256};
\draw [-stealth] (It) -- (fc);
\draw [-stealth] (fc) -- (ft);
\end{tikzpicture}
%
\caption{POMDP on the left, neural network implementation on the right. Nav A3C architecture on left and two possible Encoders: CNN encoder on center, FC encoder on right. We use CNN encoder for the 3D world while FC encoder for the gridworld.
}
\end{figure}

\paragraph{Asynchronous Advantage Actor-Critic}
\def\charelig{\nabla_{\theta_\pi}\ln \pit(\acttn; \theta_\pi)}
% There are many different variations of RL. There are many different RL algorithms: value-based methods like Q-learning and SARSA and policy-based method like actor-critic.
In this paper we use policy-based method called Asynchronous Advantage Actor-Critic (A3C) \cite{MnBaMiICML2016} that allows weight updates to happen asynchronously in a multi-threaded environment.
It works by keeping a ``shared and slowly changing copy of target network'' that is updated every few iterations by accumulated gradients in each of the threads.
The gradients are never applied to the local copy of the weights, but the local copy of weights is periodically synced from the shared copy of target weights.
The gradient for weight update is proportional to the product of \emph{advantage}, $R_t - \Value_t(\theta_\Value)$, and \emph{characteristic eligibility}, $\charelig$ \cite{WiML1992}, updating the weights according to the following update equations
\begin{align}
  \theta_\pi &\leftarrow \theta_\pi
  + \sum_{t \in \text{episode}}\alpha_\pi \charelig (R_t - \Value_t(\theta_\Value))
  \\
  \theta_\Value &\leftarrow \theta_\Value
  + \sum_{t \in \text{episode}} \alpha_\Value \frac{\partial (R_t - \Value_t(\theta_\Value))^2}
                  {\partial\theta_\Value}
                  \, .
\end{align}

For more details of the A3C algorithm please refer to \cite{MnBaMiICML2016}.
