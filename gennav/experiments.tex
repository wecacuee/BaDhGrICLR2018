Since Deep reinforcement learning algorithms need millions of iterations, in the absence of thousands of robotic replicas like \cite{LePaKrISER2017}, we evaluate the algorithms on a simulated environment.
We use the same game engine as used by \cite{MiPaViICLR2017}, called Deepmind's Lab \cite{BeLeTeARXIV2016}.
The game is setup such that an agent is placed in a 9 by 9 maze which also contains \emph{goal} at a particular location.
On reaching the goal, the agent \emph{respawns} and is free to find the goal again.
The aim of the goal is to find the goal as many times as posible in a fixed amount of time.

% Reward formulation
The setup of game is such that it enables evaluating the algorithm for optimal path planning as well as enable dividing experiments into exploration and exploitation stages.
We introduce a metric that evaluates how well the agent exploits the information gained during exploration before finding the goal for the first time.
% The optimal way of finding the goal in fastest way is to collect
% enough information for creating a map and use the map to find the shortest path from start to end location.
% However, an end-to-end navigation algorithm might find other ways of finding optimal strategies to find the goal.

Following \cite{MiPaViICLR2017} we also scatter the maze with \emph{apple} reward (+1) and assign goal with +10 reward.
The apples encourage exploration and are crucial during the traing but removing apples during evaluation does not hurt the performance in any significant way (Fig~\ref{fig:num-training-maps}).

Unlike \cite{MiPaViICLR2017}, we find small wall penality (-0.2) that pushes the agent away from the wall in very small quantities is useful to make the agent move away from moving backward and sliding along the wall as an exploration strategy discarding vision all together.
Also, we use a simple 4-action space (move forward/backward, rotate left/right) rather than an 8-action space used by \cite{MiPaViICLR2017}.
We generate 1100 random maps using depth first search followed by random connections for loop closure.
Of the 1100 maps, 1000 maps are used for training while remaining 100 are used for testing.
Also unlike \cite{MiPaViICLR2017}, we use randomly textured walls instead of fixed texture so that the policy learnt is independent of texture.
Finally, all our walls are equalantly thick to a corridor which is an artifact of how Deepmind's Lab allows the maps to be automatically generated.

Througout the experiments we will refer the goal or spawn location to random or static.
By random goal we mean that the goal position is chosen randomly for each training and testing episode, but the goal location stays constant for the episode even after the agent finds the goal and respawns.
If the goal location stays constant across training and testing episodes, we call it a static goal.
Similarly, if the spawn location stays constant through training and testing episodes, we call it a static spawn while random spawn means a different location everytime the agent respawns, either due to finding a goal or due to a new episode.

\begin{figure}[t!]%
\centering%
\def\figw{0.16\columnwidth}%
\newcommand{\includesnapshot}[1]{%
  \includegraphics[width=\figw,trim=336pt 0 0 0,clip]{#1}}%
\includesnapshot{images/snapshot/00063000-snapshot.png}%
\includesnapshot{images/snapshot/00087000-snapshot.png}%
\includesnapshot{images/snapshot/00156165-snapshot.png}%
\includesnapshot{images/snapshot/00159840-snapshot.png}%
\includesnapshot{images/snapshot/00166605-snapshot.png}%
\includegraphics[width=\figw]{images/dhiman_0002_entityLayer.pdf}\\
\includesnapshot{images/snapshot/00193065-snapshot.png}%
\includesnapshot{images/snapshot/00338220-snapshot.png}%
\includesnapshot{images/snapshot/00344985-snapshot.png}%
\includesnapshot{images/snapshot/00948930-snapshot.png}%
\includesnapshot{images/snapshot/00956325-snapshot.png}%
\includegraphics[width=\figw]{images/dhiman_0003_entityLayer.pdf}\\

\caption{The ten randomly chosen mazes for evaluation. We generate 1100 random mazes and choose ten to evaluate our experiments that require testing and training on the same maps.}%
\label{fig:environments}
\end{figure}


\subsection{Experiments}
\label{sec:navtasks}
We evaluate the Nav-A3C algorithm on maps with 5 stages of difficulty. While the Nav-A3C algorithm works smoothly on the easier stages, it does not perform better than bug-exploration methods on the hardest stage.
We propose these experiments as a 5-stage benchmark for all end-to-end navigation algorithms.

%While there already exists optimal algorithms to find shortest path between two points and perform optimal navigation between given set of points in a given map, the advantage of Deep reinforcement learning comes from its ability to extract required features from the input images and its promise
%optimize mapping and path planning in end-to-end fashion. 
%Thus we need to either integrate existing path planning and mapping methods with deep learning methods to perform end-to-end training or extend deep learning methods to learn path planning and mapping.

\begin{description}
  \ditem{Static goal, static spawn, and static map}
  \label{prob:sss}
  % VD: Ask people if the name of the experiment makes it clear
  %In this setup, we keep the goal location, spawn location and the map fixed during both training and testing.
  %This is the easiest variation of our experiments with the environment being deterministic. 
  To perform optimally on this experiment, the agent needs to find and learn the shortest path at training time and repeat it during testing. 

  \ditem{Static goal, random spawn and static map}
  % VD: Probably clear from the name
  % In this setup, we keep the goal location and the map fixed during both training and testing but chose a random spawn point every time the agents re-spawns.
  This is a textbook version of the reinforcement learning problem, especially in gridworld \cite{SuBaBOOK1998}, with the only difference being that the environment partially instead of fully observable.
  This problem is more difficult than Problem~\ref{prob:sss} because the agent
  must find an optimal policy to the goal from each possible starting point in the maze.
  \ditem{Random goal, static spawn, and static map}
  % VD: The randomness of the goal is what we need to qualify
  In this setup, we keep the spawn location and the map fixed during both training and testing but chose a random goal location for each episode.
  Note that the goal location stays constant throughout an episode.
  % VD: This line does not add any information
  %This experiments highlights the map-exploitation ability of the agent.
  The agent can perform well on this experiment, by remembering the goal location after it has been discovered and exploiting the information to revisit goal faster.  
  
  Following \cite{MiPaViICLR2017}, we report the ratio
  of time taken to hit the goal for the first time (exploration time) vs the average amount of time taken to hit goal subsequently (exploitation time). The metric, called \emph{\LatencyOneGtOne{}}, is a measure of how efficiently the agent exploits map information to find shorter path once goal location is known. 
  If this ratio is greater than 1, then we say that the agent is doing better than random exploration and higher values is better.
  \ditem{Random goal, random spawn, and static map}
  In this version of the experiment both the spawn point and the goal location is randomized. To perform optimally, the agent must localize itself within the map in addition to being able to exploit map-information.
  
  This is the problem that is addressed by \cite{MiPaViICLR2017} with limited success. 
  They evaluate this case on two maps and report \LatencyOneGtOne{} to be greater than 1 in one of the two maps. We evaluate the same metric on ten other maps and provide the tools for application to several more.
  \ditem{Random goal, random spawn, and random map}
    We believe that any proposed algorithms on end-to-end navigation problems, should be evaluated on unseen maps.
    To our knowledge, this is the first paper to do so in the case of deep reinforcement learning based navigation methods.
    We train agents to simultaneously learn to explore 1, 10, 100, 500 and 1000 maps and test them on the same 100 unseen maps. The relevant reward curves showing performance over training and testing can be found in Fig~\ref{fig:plot_reward_on_testing}. 
\end{description}

The comparative evaluation of different the stages of this benchmark are shown in Fig~\ref{fig:latency-goal-reward} and expanded upon in the next section.

%\subsection{Evaluation Maps}
%We evaluate our trained models on a few qualitative maps previously unseen during training time. Each of these maps have two paths to the goal. 
%We evaluate the percentage of times the agent travels to the goal along the shortest path after discovering it in the exploitation phase. 
%The purpose of these maps is to quantitavely evaluate whether these trained models can translate map-exploitative abilities to new, unseen maps. 
%
%\setcounter{Benchmark}{0}
%\begin{description}
%    \ditem{Square Map}
%        
%    \ditem{Wrench map}
%    This map is shown on the top row, extreme right column in Fig~\ref{fig:environments}. It has one loop and a corridor.
%    The goal is placed asymmetrically in the loop and the agent is spawned in the corridor.
%    Because of asymmetrical placement of the goal, one of the path to the goal is shorter than the other and there are only two possible paths.
%    We record the path taken each time. If the agent has learned a greedy strategy, then it would repeat the path taken for the first time.
%    We evaluate the fraction of times, the agent repeats the first path. A higher number indicates a greedy strategy.
%    In second evaluation, we let the agent explore randomly untill it hits the goal via the shorterst path.
%    At this point we evaluate the  fraction of times shortest path is taken. For this score 
%    \ditem{Goal map}
%    This map is shown on the bottom row, extreme right column in Fig~\ref{fig:environments}.
%    We chose this map because this is the simplest map with a fork. Making the map any simpler will make the map homeomorphic to a straight line. 
%    We evaluate the number of times the agent takes the right descision at the fork after exploring the goal once.
%\end{description}

\begin{figure}%
\includegraphics[width=0.5\columnwidth]{images/plot_reward_3D-1000.pdf}%
\includegraphics[width=0.5\columnwidth]{images/plot_probability_3D-1000.pdf}%
\vspace{-1em}%
\caption{Mean reward while tested on 100 unseen maps, while being trained on different number of training maps. Note that while training on 1000 maps eventually achieves high reward, it is only higher mean reward (44.2), training on 1 map hits the maximum (31) much faster.}%
\label{fig:plot_reward_on_testing}%
\end{figure}


We evaluate the Nav-A3C\cite{MiPaViICLR2017} algorithm on randomly chosen ten maps with increasing difficulty.
The results of our experiments are shown in Fig~\ref{fig:latency-goal-reward}.
We change the randomness of either the spawn point or the goal location.
We note that the increasing randomness gradually reduce the maximum reward achived and the number of times the goal is reached. As expected, the variance in reward and number of times goals is reached also increases with randomless.

\begin{figure}%
  \includegraphics[width=\linewidth]{images/plot_summary_bar_plots.pdf}%
  \vspace{-1em}%
  \caption{We evaluate the Nav-A3C\cite{MiPaViICLR2017} algorithm on randomly chosen ten maps with increasing difficulty as described in Sec~\ref{sec:navtasks}.
  Vertical axes is one of the ten maps on which the agent was trained and evaluated.
  Horizontal axes are different evaluation metrics.
  The abbreviations in legend are as follows: ``St'' stands for
  static value throughout the training and testing cycle, while ``Rnd'' means random value. ``G'', ``S'' and ``M'' stand for goal, spawn location and map respectively.
  We note that when the goal is static then rewards are consitently higher as compared to random goal while static spawn location and random spawn location are roughly close to each other within bounds of uncertainity. As expected, switching each variable from static to random increases the standard deviation on the results.
  From the \LatencyOneGtOne{} results we note that the current state of art algorithms do well when trained and tested on the same map but fail to generalize to new maps when evaluated on ability to exploit the information about goal location.}%
\label{fig:latency-goal-reward}%
\end{figure}


\subsection{Evaluation Metrics}
We report three evaluation metrics in all our experiments: reward, average number of goal hits and \LatencyOneGtOne{}. We run evaluations on a map for 100 episodes and take the mean and standard deviation of the rewards per episode and number of goal hits per episode. Even though reward includes rewards due to apples and negative rewards due to wall penalities, the reward due to goal dominates dominates the reward metric making reward approximately ten times the average goal hits.

The \LatencyOneGtOne{} is defined as the ratio of time take to find the goal first time to the average time take to find goal thereafter.
Note that whenever the goal location is ``Random'' (different in each episode but same for the episode), untill the goal is found for the first time, the agent is just exploring the map.
After the first goal hit, the agent should be able to exploit the location of the goal to find the goal in shorter times.
Hence a \LatencyOneGtOne{} value greater than one indicates that the algorithm is able to successfully exploit information gathered during goal finding exploration.



