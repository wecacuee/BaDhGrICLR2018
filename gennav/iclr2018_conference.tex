\documentclass{article} % For LaTeX2e
\usepackage{iclr2018_conference,times}
\usepackage{hyperref}
\usepackage{url}
\usepackage{amsfonts}
\usepackage{amsmath}
\usepackage{comment}
\usepackage{float}
\usepackage{import}
\usepackage{multirow}
\usepackage{subfigure}
\usepackage{tikz}
\usetikzlibrary{positioning}
\usepackage[inline]{enumitem}
\usepackage{amsthm}
\graphicspath{ {images/} }
\usepackage{soul}

%%%%%%%%% Macros %%%%%%%%%%%%%%%%%%
\newtheorem{defn}{Definition}
\newif\ifblind
\def\goldenr{1.618}
\newcommand*\textfrac[2]{
      \frac{\text{#1}}{\text{#2}}
  }
\newcommand{\TODO}[1]{TODO:{#1}}
\newcommand{\etal}{et~al.}
\def\state{s}
\def\statet{\state_t}
\def\statetp{\state_{t-1}}
\def\statetn{\state_{t+1}}
\def\obs{o}
\def\obst{\obs_t}
\def\act{a}
\def\actt{\act_t}
\def\acttp{\act_{t-1}}
\def\acttn{\act_{t+1}}
\def\Obs{\mathcal{O}}
\def\ObsFunc{C}
\def\ObsFuncFull{\ObsFunc(\statet, \actt) \rightarrow \obst}
\def\ObsFuncInv{\ObsFunc^{-1}}
\def\ObsFuncInvFull{\ObsFuncInv(\obst, \statetp, \actt) \rightarrow \statet}
\def\State{\mathcal{S}}
\def\Action{\mathcal{A}}
\def\Trans{T}
\def\TransFull{\Trans(\statet, \actt) \rightarrow \statetn}
\def\TransObs{T_c}
\def\Rew{R}
\def\rew{r}
\def\rewt{\rew_t}
\def\rewtp{\rew_{t-1}}
\def\rewtn{\rew_{t+1}}
\def\RewFull{\Rew(\statet, \actt) \rightarrow \rewtn}
\def\TransObsFull{\TransObs(\statet, \obst, \actt, \rewt; \theta_T) \rightarrow \statetn}
\def\Value{V}
\def\pit{\pi_t}
\def\piDef{\pi(\acttn|\statet, \obst, \actt, \rewt; \theta_\pi) \rightarrow \pit(\acttn ; \theta_\pi)}
\def\Valuet{\Value_t}
\def\ValueDef{\Value(\statet, \obst, \actt, \rewt; \theta_\Value) \rightarrow \Valuet(\theta_\Value)}
\def\R{\mathbb{R}}
\def\E{\mathbb{E}}

\title{BLINC: Querying the Internal Belief States of Deep-Reinforcement Learning Methods}

% Authors must not appear in the submitted version. They should be hidden
% as long as the \iclrfinalcopy macro remains commented out below.
% Non-anonymous submissions will be rejected without review.

\author{Shurjo Banerjee*, Vikas Dhiman*, Brent Griffin \& Jason J. Corso \thanks{indicates equal contribution} \\
The Electrical Engineering and Computer Science Deparment\\
The University of Michigan\\
Ann Arbor, MI 48109, USA \\
\texttt{\{shurjo,dhiman,griffb,jjcorso\}@umich.edu} \\
}

% The \author macro works with any number of authors. There are two commands
% used to separate the names and addresses of multiple authors: \And and \AND.
%
% Using \And between authors leaves it to \LaTeX{} to determine where to break
% the lines. Using \AND forces a linebreak at that point. So, if \LaTeX{}
% puts 3 of 4 authors names on the first line, and the last on the second
% line, try using \AND instead of \And before the third author name.

\newcommand{\fix}{\marginpar{FIX}}
\newcommand{\new}{\marginpar{NEW}}

\iclrfinalcopy % Uncomment for camera-ready version, but NOT for submission.

\begin{document}


\maketitle

\begin{abstract}
    Deep reinforcement learning (DRL) algorithms have demonstrated strong progress in learning to reach a goal in challenging three-dimensional environments without requiring any explicit SLAM or path-planning in their navigation. While promising, the limitations and underlying pattern recognitions performed by these networks based approaches are not very well understood. In this work we introduce the BLINC training paradigm - an appendum to standard DRL models that can be used to (a) Fine-tune DRL algorithms and improve their performance (b) Afford new mechanics allowing for the querying of the internal states of these methods (more details as I figure it out). In BLINC, agents are incentivized to blind themselves as often as possible in the course of their navigation so as to more naturally pick up on aspects of long term planning. We find that BLINC consistently improves rewards scores by 5\% across multiple environments and worlds. Our querying mechanics provide both qualitative and quantiative notions of the kind of beliefs learned and propagated by the underlying layers of these network based agents. 
\end{abstract}

\section{Introduction}
%% Outline
% 1. Navigation is important problem
% 1.5 Traditionally addressed by mapping during exploration and path
%      planning during exploitation.
% 2. End to end learning algorithms have shown promise to take over
%      mapping and path
% 3. We do not know how these algorithms work. There has been work in computer vision that shows the learning on neural network based methods can be learning totally different kind of patterns from what we would expect.
% 4 We introduce the concept of blinding to force the learning of of long term planning. We showcase improved scores over industry standard baselines.
% 5 We showcase that blinding affords an understand of the implicit abilities of these DRL agents.

% 1. Navigation is important problem
% 1.5 Traditionally addressed by mapping during exploration and path
%      planning during exploitation.
Navigation remains a fundamental problems in mobile robotics and artificial intelligence~\cite{SmChIJRR1986,ElCOMPUTER1980}.
The problem, traditionally called SLAM (Simulataneous Localization and Mapping), is classically addressed by separating the eventual task of navigation into \textit{exploration} and \textit{exploitation}. In the exploration phase, the environment is incrementally built and represented in some sort of \emph{map} data-structure. In exploitation, this data structure is used for localization and path-planning to find an optimal path to a given destination based on desired optimality criterion.  Although there have been many advances in this classical approach \cite{XXX}, it remains a difficult challenge. \textit{Either mention some examples or cite a paper that highlights the failiures of SLAM or both!} 

%
% 2. End to end learning algorithms have shown promise to take over
%      mapping and path-planning
More recently, end-to-end navigation methods---methods that attempt to  
solve the navigation problem without breaking it down into the separate parts of localization, mapping and path-planning---have gained traction.
With the recent success of Deep Reinforcement Learning (\textbf{DRL}) \cite{MnKaSiNATURE2015,MnKaSiNATURE2015}, these end-to-end navigation methods \cite{MnBaMiICML2016,SiHuMaNATURE2016,LePaKrISER2017,MiPaViICLR2017,OhChSiICML2016} forego decisions about the details that are required in the intermediate step of map building.  

Work by Mirowski \etal{} showcased agents that learned to navigate textureless environments to find desired goal locations trained on pure monocular vision - a feat that is still quite difficult for state-of-the-art monocular SLAM systems \cite{XXX}. \textit{Talk about the memory structures used - no need for any explicit path planning, slam or all that nonsense} 
The potential for simpler yet capable methods is rich on the surface.

% 3. We do not know how these algorithms work. There has been work in computer vision that shows the learning on neural network based methods can be learning totally different kind of patterns from what we would expect.
Despite this potential and recent successes, state-of-the-art DRL based methods have been confronted with their own set of problems. In line with other Deep-Learning fallacies (\textit{too negative?}), foremost among these is the difficulty in understanding the method limitations or the kind of patterns that these algorithms are understanding. The inherent black-box nature of these methods make them hard to study. 

% 4.1 We find that it is not remembering the map it is being trained on
% 4.2 We find that no path planning is  happening only, memorizing and regeneration of the sequence of steps. However, it is not 
In this work, we attempt to pull back the lid of how these networks appear to be in fact be performing this navigation. We phrase these queries within the context of exploration and exploitation as is traditional in the SLAM world. Our contributions are three-fold:
\begin{enumerate}
\item We succesively blind state-of-the art DRL agents in a curriculum fashion to gain an understanding of whether these agents can be forced to perform long-term planning in the execution of their learned navigation strategies.
\item In a bid to more easily teach agents to perform long-term planning, we introduce BLINC. BLINC, is a conceptually simple modification applicable to all DRL methods wherein agents are incentivized to blind themselves during navigation as often as possible. Extra  incentives are provided when this blindness is contiguously performed over several frames.  We showcase how agents trained via BLINC acheive better performance then current state-of-the-art methods. 
\item We showcase BLINCs great strength in affording the ability to easily query the hidden states of the networks used by these models. We use this method to gain an understanding of each agent's explicit understanding of its surroundings at given points of time.
\end {enumerate}



\section{Related Work}
\paragraph{Localization and mapping}
Robotic localization and mapping for navigation as a problem since the beginning of mobile robotics and sensing.
Smith and Cheeseman~\cite{SmChIJRR1986} introduced the idea of propagating spatial uncertainty for robot localization while mapping and Elfes popularized Occupancy Grids~\cite{ElCOMPUTER1980} for mapping.
In the last three decades, the field has exploded with variation of algorithms for different sensors like cameras, laser scanners, sonars, depth sensors, variation in level of detail like topological maps \cite{KuCOGSCI1978} for low level of detail to occupancy grid maps for high detail and variation in environment types like highly textured or non-textured.

All these approaches require huge amount of hand-tuning and design for adapting to different environments and sensor types. The level of detail of maps also needs to be decided before hand irrespective of the application and hence is not optimized for the application at hand.

\paragraph{Deep reinforcement learning}
Deep reinforcement learning (DRL) came back to the limelight \cite{TeACM1995} \hl{Check whether this citation should be here} with Mnih \etal~ \cite{MnKaSiNATURE2015,MnKaSiNIPSDLW2013} demonstrating that their algorithms outperform humans on Atari games. Subsequently, the DRL algorithms have been extended \cite{MnBaMiICML2016} and applied to various games \cite{SiHuMaNATURE2016}, simulated platforms \cite{KaStJoNIPS2017}, real world robots \cite{LePaKrISER2017} and more recently to robotic navigation \cite{MiPaViICLR2017,OhChSiICML2016}.

The exploration into robotic navigation using deep reinforcement learning is a nascent topic, it has potential to disrupt the fields of simultaneous localization and mapping and path planning. Also, \cite{MiPaViICLR2017} train and test on the same maps which limits our understanding of the generality of the method. In fact, it is very common to train and test on the same environments in reinforcement learning based navigation works \cite{zhu2017target,kulkarni2016deep} with the only variation being in location of goal and starting point. In contrast, \cite{OhChSiICML2016} do test on random maps but the only decision that the agent has to make is avoid a goal of particular color and seek other color rather than remembering the path to the goal. On similar lines, \cite{chaplottransfer} test their method on unseen maps in VizDoom environment but only vary the maps by unseen texture.
%
In this work, we take the study of these methods significantly farther with a thorough investigation of whether DRL-based agents remember enough information to obviate mapping algorithms or the need to be augmented with mapping algorithms.


\section{Background}
Our experimental setup is inspired by Mirowski \etal{} work \cite{MiPaViICLR2017}. We summarize the technical setup here for completeness. We recommend \cite{MnBaMiICML2016,MnBaMiICML2016,MiPaViICLR2017}

The problem of navigation is formulated as interaction between environment and agent. At time time $t$ the agent takes an action $\actt \in \Action$ and observes observation $\obst \in \Obs$ along with a real reward $\rewt \in \R$.
We assume the environment to be Partially Observable Markov Decision Process (POMDP).
In a POMDP the state of the environment $\statet \in \State$ is assumed to be the only information that is propagated over time and both $\obst$ and $\rewt$ are assumed to be independent of previous states given current state and last action. Formally, a POMDP is a six tuple $(\Obs, \ObsFunc, \State, \Action, \Trans, \Rew)$ that is observation space $\Obs$, observation function $\ObsFuncFull$, state space $\State$, action space $\Action$, transition function $\TransFull$ and reward function $\RewFull$ respectively.
For our problem setup, the observation space $\Obs$ is the space of encoded feature vector that can be generated from input image or combination of other inputs, action space $\Action$ contains four actions: rotate left, rotate right, moved forward and move backward and reward function $\Rew$ is defined for each experiment so that the reaching the goal leads to high reward with auxilary reward to encourage certain kind of behavior.

For Deep reinforement learning the state space $\State$ is not hand tuned, instead it is modeled as semantically meaningless \emph{hidden state} of a fixed size float vector.
Also, instead of modeling observation function $\ObsFuncFull$ and $\TransFull$, a combined transition function $\TransObsFull$ is modeled such that it estimates next state $\statetn$ directly considering previous observation as well as reward into account. For policy-based DRL a policy function $\piDef$ and a value function $\ValueDef$ are also modeled. All three functions $\TransObs$, $\pit$, $\Valuet$ share most of the parameters in a way such that $\theta_T \subseteq \theta_{\pi} \cap \theta_\Value$

%Since we aim to estimate $\ObsFuncInv$ and $\Trans$ from experience, we formulate the experience as observation tuples divided into episodes of fixed length $E$.
%Each episode experience contains $E$ tuples with observation, action and corresponding reward $D_E = \{(\obs_0, \act_0, r_0), \dots, (\obs_E, \act_E, r_E)\}$. After each episode the state $\state_t$ is reset to all zeros and another data sequence is collected. Let the collected dataset be $D_N = \{
Our objective is to estimate unknown weights $\theta = \theta_T \cup \theta_\pi \cup \theta_V$ that maximizes the expected future reward, $R_t = \sum_{k=t}^{t_{end} - t} \gamma^{k-t} r_k$, where $\gamma$ is the discount factor,
%
\begin{align}
\theta^* = \arg\max_{\theta} \E[R_t] \,.
\end{align}
%
% need \graphicspath{{images/}}
\def\svgwidth{0.5\columnwidth}%
\begin{figure}%
\input{images/a3c-as-pomdp.pdf_tex}%
\def\svgwidth{0.5\columnwidth}%
\input{images/a3c-as-nn.pdf_tex}
\caption{POMDP on the left, neural network implementation on the right.}
\end{figure}

\paragraph{Asynchronous Advantage Actor-Critic}
\def\charelig{\nabla_{\theta_\pi}\ln \pit(\acttn; \theta_\pi)}
% There are many different variations of RL. There are many different RL algorithms: value-based methods like Q-learning and SARSA and policy-based method like actor-critic.
In this paper we use policy-based method called Asynchronous Advantage Actor-Critic (A3C) \cite{MnBaMiICML2016} that allows weight updates to happen asynchronously in a multi-threaded environment.
It works by keeping a ``shared and slowly changing copy of target network'' that is updated every few iterations by accumulated gradients in each of the threads.
The gradients are never applied to the local copy of the weights, but the local copy of weights is periodically synced from the shared copy of target weights.
The gradient for weight update is proportional to the product of \emph{advantage}, $R_t - \Value_t(\theta_\Value)$, and \emph{characteristic eligibility}, $\charelig$ \cite{WiML1992}, updating the weights according to the following update equations
\begin{align}
  \theta_\pi &\leftarrow \theta_\pi
  + \sum_{t \in \text{episode}}\alpha_\pi \charelig (R_t - \Value_t(\theta_\Value))
  \\
  \theta_\Value &\leftarrow \theta_\Value
  + \sum_{t \in \text{episode}} \alpha_\Value \frac{\partial (R_t - \Value_t(\theta_\Value))^2}
                  {\partial\theta_\Value}
                  \, .
\end{align}

For more details of the A3C algorithm please refer to \cite{MnBaMiICML2016}.


\section{Approach}
% 0. Describe baseline models
% 1. Introduce blinding in a curriculum and finetuning manner
% 2. Introduce the concept of doubling the action space
% 3. Introduce the potential applicablity of this blinding to querying
% the surrounding states of the agent.

\subsection{Baseline Models}

\subsection{Blinding: Curriculum and Finetuning}

\subsection{Blinding: Self-Supervised Curriculum Training}


\section{Experiments}
% 0. Baseline models
% 1. Blinding is easy enough to understand
% 2. The doubling of the action space - that's the brilliance
%

\subsection{Baseline Models}
\begin{table}[h]
    \label{sample-table}
    \begin{center}
        \begin{tabular}{ll}
            \multicolumn{1}{c}{\bf PART}  &\multicolumn{1}{c}{\bf DESCRIPTION}
            \\ \hline \\
            Dendrite         &Input terminal \\
            Axon             &Output terminal \\
            Soma             &Cell body (contains cell nucleus) \\
        \end{tabular}
    \end{center}
    \caption{Baseline experiments.}
\end{table}


\subsection{Blinding: Curriculum and Finetuning}

\subsection{Blinding: Self-Supervised Curriculum Training}



\section{Analysis}



\section{Conclusion}
\input{blinc-conclusion}


\subsubsection*{Acknowledgments}

Use unnumbered third level headings for the acknowledgments. All
acknowledgments, including those to funding agencies, go at the end of the paper.

\bibliography{iclr2018_conference}
\bibliography{shared}
\bibliographystyle{iclr2018_conference}

\end{document}
