Deep reinforcement learning (DRL) algorithms have demonstrated strong progress in learning to reach a goal in challenging three-dimensional environments without requiring any explicit SLAM or path-planning in their navigation. 
While promising, the limitations and underlying pattern recognitions performed by these networks based approaches are not very well understood. 
In this work we introduce the BLINC training paradigm - an appendum to standard DRL models that can be used to (a) Fine-tune DRL algorithms and improve their performance (b) Afford new mechanics allowing for the querying of the internal states of these methods. 
In BLINC, agents are incentivized to blind themselves as often as possible in the course of their navigation so as to more naturally lock on to aspects of long-term planning in the context of their motion. 
We find that BLINC consistently improves rewards scores by x\% across multiple environments and worlds. Our querying mechanics provide both qualitative and quantiative notions of the kind of beliefs learned and propagated by the underlying layers of these network based agents. 

